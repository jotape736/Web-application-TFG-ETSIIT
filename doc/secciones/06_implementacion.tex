\chapter{Implementación}

En esta sección se describen las diferentes etapas que se han llevado a cabo durante el desarrollo del proyecto, los problemas encontrados y las soluciones propuestas. Se incluyen esquemas, capturas de código y ejemplos de uso de la aplicación.\newpage

\section{Preparación del entorno de desarrollo}

Previo a la implementación, se realizó la configuración del entorno de desarrollo. Esto incluye la instalación de todas las dependencias necesarias, así como las herramientas mencionadas en la sección anterior.\newline

\section{Servidor}
En este apartado se describe la implementación del servidor de la plataforma. Se especificará la estructura del código así como la implementación de la API mencionada en la sección de diseño.\newline

FALTA ESTO
\begin{enumerate}
    \item Implementación del Modelo:
    \item Implementación de Esquemas:
\end{enumerate}

\subsection{Servicios}

El servidor responde a solicitudes de creación y consulta de entidades. Estas operaciones pueden ser llevadas a cabo tanto por usuarios genéricos como por administradores.

\subsubsection*{Algoritmo de Calendarios}

El algoritmo empleado para la creación de las combinaciones de grupos para los calendarios fue propuesto por \textbf{Jesús García Miranda}, secretario de la Escuela Técnica Superior de Ingenierías Informática y de Telecomunicaciones de Universidad de Granada con dirección de correo electrónico \href{mailto:jesusgm@ugr.es}{jesusgm@ugr.es}. Dada la complejidad de los algoritmos presentados en la sección 2, se optó por una solución que permitiera la creación de calendarios de forma rápida y sencilla.\newline

Antes de mostrar el pseudocódigo del algoritmo, téngase en cuenta que las horas de los grupos de las asignaturas se almacenan codificadas de la siguiente forma:\newline

El \textbf{día} de la semana se indica mediante las \textbf{centenas}, la \textbf{hora de inicio} de la clase se indica por medio de las \textbf{decenas y unidades}. Al ser 5 los días lectivos de la semana y 13 las horas que van desde el inicio de la jornada (8:30) hasta el final (20:00), una clase de 2 horas el lunes de 8:30 a 10:30 se codificaría como los valores 101 y 102.\newline


\begin{algorithm}
    \caption{Generar Horarios}
    \label{alg:generar_horarios}
    \begin{algorithmic}[1]
        \Require Conjunto de asignaturas con sus respectivos subgrupos
        \Ensure Lista de los 10 mejores horarios posibles
        \Function{GenerarHorarios}{asignaturas}
            \State Combinar los subgrupos seleccionados utilizando el producto cartesiano
            \State Crear un diccionario que almacene las combinaciones
            \For{cada combinación en los productos}
                \State Inicializar una lista vacía para almacenar las horas de clase
                \For{cada asignatura en la combinación}
                    \State Obtener las horas de teoría para el subgrupo.
                    \State Obtener las horas de prácticas para el subgrupo
                \EndFor
                \State Ordenar la lista de horas y almacenarla en el diccionario de horarios
            \EndFor
            \State Inicializar una lista vacía para almacenar los horarios evaluados
            \For{cada horario en el diccionario de horarios}
                \State Contar las horas repetidas para determinar los solapamientos
                \State Contar los días de asistencia
                \State Calcular las horas libres entre las clases agrupando las horas
                \State por centenas y contando los números que faltan en la serie
                \State Añadir el horario y sus ponderaciones a la lista de horarios evaluados
            \EndFor
            \State Ordenar los horarios por menor número de solapamientos y horas libres
            \State Tomar los 10 mejores horarios y almacenarlos en la lista de resultados
            \Return Lista de los 10 mejores horarios
        \EndFunction
    \end{algorithmic}
\end{algorithm}

\section{Cliente}

El cliente es la parte con la que el usuario final podrá interactuar. Hace de interfaz para las peticiones a la API resueltas por el servidor, que es donde se encuentra toda la lógica.\newline

La implementación de cliente se ha llevado a cabo utilizando la librería de React, que permite la creación de interfaces de usuario de forma sencilla y eficiente. La estructura se divide en componentes, que son elementos independientes que se encargan de mostrar la información y gestionar las interacciones con el usuario. Dichos componentes son la base de React y permiten la creación de aplicaciones escalables y fáciles de mantener.\newline

Todo este proceso se explica en detalle a continuación:\newline

\subsection{Componentes de React}

Son piezas de código reutilizable que renderizan una parte de la interfaz, pueden ser parametrizados, por medio de entradas llamadas ``props'' y pueden contener su propio estado. El estado es un objeto que contiene datos que pueden cambiar y se usan para controlar los cambios en la interfaz. Los componentes permiten componer interfaces más complejas a partir de elementos más sencillos, lo que facilita la escalabilidad y el mantenimiento.\newline


Para el desarrollo de la plataforma \textbf{GAC} se han desarrollado los siguientes componentes:

\begin{enumerate}
    \item \textbf{YearPanel}: Panel que muestra los cursos y cuatrimestres del grado seleccionado.
    \begin{enumerate}
        \item \textbf{SelectorQuarter}: Selector que permite elegir el cuatrimestre.
        \item \textbf{SelectorYear}: Selector que permite elegir el curso.
    \end{enumerate}
    \item \textbf{SubjectPanel}: Panel que muestra las asignaturas del curso seleccionado.
    \item \textbf{GroupPanel}: Panel que muestra los grupos de las asignaturas seleccionadas.
    \item \textbf{Calendar}: Cuadro que muestra los horarios de las asignaturas y grupos seleccionados.
\end{enumerate}

\subsection{Integración con Backend}
