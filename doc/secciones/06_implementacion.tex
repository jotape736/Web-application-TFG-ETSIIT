\chapter{Implementación}

En esta sección se describen las diferentes etapas que se han llevado a cabo durante el desarrollo del proyecto, los problemas encontrados y las soluciones propuestas. Se incluyen esquemas, capturas de código y ejemplos de uso de la plataforma \textbf{GAC}.\newpage

\section{Preparación del entorno de desarrollo}

Previo a la implementación, se realizó la configuración del entorno de desarrollo. Esto incluye la instalación de todas las dependencias necesarias, así como las herramientas mencionadas en la sección anterior.\newline

\section{Backend}
En este apartado se describe la implementación del servicio Backend de la plataforma. Se especificará la estructura del código así como la implementación de la API mencionada en la sección de diseño.\newline

FALTA ESTO
\begin{enumerate}
    \item Implementación del Modelo:
    \item Implementación de Esquemas:
\end{enumerate}

\subsection{Servicios}

El backend responde a solicitudes de creación y consulta de entidades. Estas operaciones pueden ser llevadas a cabo tanto por usuarios genéricos como por administradores.

\subsubsection*{Algoritmo de Calendarios}

El algoritmo empleado para la creación de las combinaciones de grupos para los calendarios fue propuesto por \textbf{Jesús García Miranda}, secretario de la \textit{Escuela Técnica Superior de Ingenierías Informática y de Telecomunicaciones de la Universidad de Granada} con dirección de correo electrónico \href{mailto:jesusgm@ugr.es}{jesusgm@ugr.es}. Dada la complejidad de los algoritmos presentados en la sección 2, se optó por una solución que permitiera la creación de calendarios de forma rápida y sencilla.\newline

Antes de mostrar el pseudocódigo del algoritmo, téngase en cuenta que las horas de los grupos de las asignaturas se almacenan codificadas de la siguiente forma:\newline

El \textbf{día} de la semana se indica mediante las \textbf{centenas}, la \textbf{hora de inicio} de la clase se indica por medio de las \textbf{decenas y unidades}. Al ser 5 los días lectivos de la semana y 13 las horas que van desde el inicio de la jornada (8:30) hasta el final (20:00), una clase de 2 horas el lunes de 8:30 a 10:30 se codificaría como la secuencia de valores [101,102].\newline

\begin{algorithm}
    \caption{Generar Horarios}
    \label{alg:generar_horarios}
    \begin{algorithmic}[1]
        \Require Diccionario de asignaturas con sus subgrupos.
        \Ensure Lista de los 10 mejores horarios posibles.
        \Function{GenerarHorarios}{asignaturas}
            \State $productos$ = $asignaturas.prodCartesiano()$
            \State Concatear el código de asignatura a cada subgrupo de cada combinación.
            \State $diccionario-combinaciones$ = $productos$
            \State $horarios$ = $dict$
            \For{$combinacion$ en $productos$}
                \State $lista-horas$ = $[]$
                \For{$asignatura$ en $combinacion$}
                    \State $lista-horas$ = $get-teoria(asignatura)$
                    \State $lista-horas$ = $lista-horas$ $\cup$ $get-practicas(asignatura)$
                \EndFor
                \State $lista-horas.sort()$ \Comment{Se ordenan los números de las horas.}
                \State $horarios.append(lista-horas)$ \Comment{Cada horario está identificado.}
            \EndFor
            \State $horarios-evaluados$ = $dict$
            \For{$horario$ en $horarios$}
                \State $Key$ = $horario.getKey()$    \Comment{Se obtiene el identificador del horario.}
                \State $ponderacion$ = $[]$
                \State $ponderacion.append(contar-solapamientos(horario))$
                \State $ponderacion.append(contar-horas-muertas(horario))$
                \State $ponderacion.append(contar-dias-asistencia(horario))$
                \State $horarios-evaluados[Key]$ = $ponderacion$
            \EndFor
            \State $resultado$ = $[]$ 
            \State $horarios-evaluados.sort()$ \Comment{Se ordenan los por ponderación.}
            \For {$i$ en $range(10)$}
                \State $resultado.append(horarios[horarios-evaluados[i]])$
            \EndFor

            \Return $resultado$
        \EndFunction
    \end{algorithmic}
\end{algorithm}

\section{Frontend}

El frontend es la parte con la que el usuario final podrá interactuar. Hace de interfaz para las peticiones a la API resueltas por el backend, que es donde se encuentra toda la lógica.\newline

La implementación del mismo se ha llevado a cabo utilizando la librería de React, que permite la creación de interfaces de usuario de forma sencilla y eficiente. La estructura se divide en componentes, que son elementos independientes que se encargan de mostrar la información y gestionar las interacciones con el usuario. Dichos componentes son la base de React y permiten la creación de aplicaciones escalables y fáciles de mantener.\newline

Todo este proceso se explica en detalle a continuación:\newline

\subsection{Componentes de React}

Son piezas de código reutilizable que renderizan una parte de la interfaz, pueden ser parametrizados, por medio de entradas llamadas ``props'' y pueden contener su propio estado. El estado es un objeto que contiene datos que pueden cambiar y se usan para controlar los cambios en la interfaz. Los componentes permiten componer interfaces más complejas a partir de elementos más sencillos, lo que facilita la escalabilidad y el mantenimiento.\newline


Para el desarrollo de la plataforma \textbf{GAC} se han desarrollado los siguientes componentes:

\begin{enumerate}
    \item \textbf{YearPanel}: Panel que muestra los cursos y cuatrimestres del grado seleccionado.
    \begin{enumerate}
        \item \textbf{SelectorQuarter}: Selector que permite elegir el cuatrimestre.
        \item \textbf{SelectorYear}: Selector que permite elegir el curso.
    \end{enumerate}
    \item \textbf{SubjectPanel}: Panel que muestra las asignaturas del curso seleccionado.
    \item \textbf{GroupPanel}: Panel que muestra los grupos de las asignaturas seleccionadas.
    \item \textbf{Calendar}: Cuadro que muestra los horarios de las asignaturas y grupos seleccionados.
\end{enumerate}

\subsection{Integración con Backend}

section{Tecnologías Utilizadas}

Para el desarrollo de la plataforma se compone de un backend y un frontend. El backend hace de servidor y es el que se encarga de responder a las peticiones por medio de una API desarrollada en Python con Flask y SQLite como base de datos. El frontend incluye la interfaz de usuario y ha sido desarrollada en ReactJS. A continuación se detallan las tecnologías utilizadas en detalle.

\subsection{Desarrollo Backend}

Para el desarrollo del backend se ha utilizado una arquitecura cliente-servidor basada en una arquitectura orientada a servicios (SOA).\newline

El backend es el encargado de administrar la funcionalidad general de la plataforma, operando en el servidor para procesar las solicitudes, manipular los datos y devolver la respuesta al cliente. En esta capa se almacena el código encargado de la lógica de negocio y la interacción con la base de datos. Su funcionalidad juega un papel esencial en la comunicación con el frontend y se asegura de que la información devuelta sea correcta, asegurando la integridad de los datos, el dinamismo y el diseño responsive de la interfaz de usuario. Actuando como núcleo funcional de la plataforma, se compone de las tecnologías que se detallan a continuación.

\subsubsection*{Python}
Es un lenguaje de programación interpretado, orientado a objetos y de alto nivel con semántica dinámica. Su sintaxis es clara y legible, lo que facilita la escritura de código y la lectura del mismo. Es un lenguaje multiplataforma, lo que significa que se puede ejecutar en cualquier sistema operativo. Es versátil y se puede utilizar en una amplia variedad de aplicaciones, desde desarrollo web hasta análisis de datos y aprendizaje automático. Es un lenguaje de programación muy popular y cuenta con una gran cantidad de bibliotecas y marcos de trabajo que facilitan el desarrollo de aplicaciones \cite{python2021python}.\newline

El motivo de usar Python para el desarrollo del backend se debe a una elección personal, motivada por el interés de aprender un nuevo lenguaje de programación tan ampliamente utilizado a nivel mundial. Python es reconocido por su simplicidad y facilidad de uso, lo que lo convierte en una excelente opción para el desarrollo de aplicaciones web. Cuenta con una gran cantidad de bibliotecas y marcos de trabajo que facilitan el desarrollo de aplicaciones web, la documentación es extensa y la comunidad es muy activa y además soporta módulos y paquetes que lo que resulta en productos modulares y reutilizables.

\subsubsection*{Flask}
Es un framework de aplicaciones web escrito en Python. Se compone de un núcleo WSGI\footnote{\url{https://wsgi.readthedocs.io/en/latest/}} simple y fácil de extender que permite a los desarrolladores crear aplicaciones web rápidamente con una mínima configuración. Flask\footnote{\url{https://pythonbasics.org/what-is-flask-python/}} es ligero y fácil de usar, lo que lo convierte en una muy buena opción para desarrollar proyectos de pequeños o medianos. Es muy popular y cuenta con una gran cantidad de bibliotecas y extensiones que facilitan el desarrollo \cite{grinberg2018flask}.\newline

La recomendación de usar Flask vino por parte de mi tutor, ya que es un framework muy popular, modular y fácil de usar. En la red hay disponibles una gran cantidad de guías que explican los pasos necesarios para hacer una configuración inicial y empezar a desarrollar aplicaciones. Dada la familiaridad con este framework y su baja curva de aprendizaje, se decidió utilizar para el desarrollo del backend.

\subsubsection*{SQLite}

SQLite\footnote{\url{https://www.sqlite.org/index.html}} es una base de datos relacional embebida, de código abierto, que se implementa como una biblioteca de programación C. Es rápida, ligera y fácil de usar, lo que la convierte en una excelente opción para aplicaciones de pequeña o mediana envergadura. Es muy popular y cuenta con una gran cantidad de bibliotecas y extensiones que facilitan el desarrollo \cite{kreibich2010using}.\newline

La elección de SQLite como base de datos se debe a que es una base de datos embebida, lo que significa que no requiere un servidor de base de datos separado para funcionar. Esto simplifica la configuración y el despliegue de la plataforma. Además, es de tipo relacional, y dada la estructura rígida de asignaturas, grupos, subgrupos, aulas, etc., se ajusta perfectamente a las necesidades de la plataforma. Por último, es muy fácil de usar y cuenta con una gran cantidad de bibliotecas y extensiones que facilitan el desarrollo, como es el caso de \textbf{SQLAlchemy}\footnote{\url{https://www.sqlalchemy.org/}} que usaremos para facilitar la interacción con la base de datos.

\subsubsection*{Docker}

Docker \footnote{\url{https://www.docker.com/}} es una tecnología de organización de contenedores que permite empaquetar aplicaciones en contenedores virtuales que pueden ejecutarse en cualquier lugar. Provee un tipo de virtualización más ligera que las máquinas virtuales tradicionales, ya que comparte el núcleo del sistema operativo con otros contenedores.\newline

El motivo de usar Docker es mi familiaridad con la herramienta y mi interés en aprender más sobre ella. Docker es una tecnología muy popular y ampliamente utilizada en la industria. Permite empaquetar aplicaciones en contenedores virtuales que pueden ejecutarse en las mismas condiciones independientemente del sistema operativo subyacente, lo que facilita la configuración y el despliegue de la plataforma. Además, es muy eficiente y ligero, permitiendo ejecutar múltiples contenedores en un mismo servidor.

\subsection{Desarrollo Frontend}

El frontend es la parte de la plataforma con la que interactúan los usuarios finales. Es la capa de presentación e incluye todo aquello que el se ve. Juega un papel fundamental en la experiencia ya que es la primera impresión que se lleva el usuario y debe ser clara e intuitiva.\newline

\subsubsection*{ReactJS}

ReactJS\footnote{\url{https://es.reactjs.org/}} es una biblioteca de JavaScript de código abierto diseñada para crear interfaces de usuario interactivas, modulares y reutilizables. Permite el desarrollo de aplicaciones web que cambian sus datos sin necesidad de recargar la página. Provee un DOM mucho más eficiente y ligero almacenado en memoria y con el que interactúa en lugar de hacerlo directamente con el DOM del navegador. En la mayoría de webframeworks, se manipula el DOM completamente en cada uno de los eventos que desencadena la página, como consecuencia de esto, en los casos en los que se modifica una gran cantidad de datos, el rendimiento se ve seriamente afectado. ReactJS soluciona este problema, ya que solo modifica los elementos que han cambiado, haciendo uso de lo que se conoce como Virtual DOM\footnote{\url{https://es.reactjs.org/docs/faq-internals.html}}. Es muy popular y cuenta con una gran cantidad de bibliotecas y extensiones que facilitan el desarrollo \cite{aggarwal2018modern}.\newline

La utilización de ReactJS se debe a que es una biblioteca de JavaScript muy popular y ampliamente utilizada en la industria. Es muy eficiente y permite el desarrollo de aplicaciones web interactivas y modulares. Además, sirve de base para el desarrollo de aplicaciones móviles con React Native\footnote{\url{https://reactnative.dev/}}.\newline

Para ilustrar su popularidad, a continuación se muestra un gráfico con las tendencias de NPM\footnote{\url{https://npmtrends.com/@angular/core-vs-angular-vs-react-vs-vue}} que pese a no ser una fuente totalmente fiable (podemos ver un pico de Vue en la gráfica que puede haberse hecho por error o maliciosamente para alterar las estadísticas), nos da una idea de la popularidad de ReactJS en la actualidad.

\begin{figure}[H]
    \centering
    \includegraphics[width=1\textwidth]{./imagenes/Comparativa.png}
    \caption{Comparativa de descargas de frameworks de acuerdo a NPM.}
\end{figure}

\newpage

\subsection{Herramientas de Desarrollo}

\renewcommand{\icon}[1]{\includegraphics[height=18pt]{#1}}
\subsubsection*{Visual Studio Code \protect\icon{./imagenes/vscode_logo.png}}

Visual Studio Code\footnote{\url{https://code.visualstudio.com/}} es un editor de código fuente desarrollado por Microsoft para Windows, Linux y macOS. Dispone de características como resaltado de sintaxis, finalización de código, refactorización de código, depuración, control de versiones, entre otras. Soporta una gran cantidad de lenguajes de programación y cuenta con una gran cantidad de extensiones que facilitan el desarrollo. \newline

La razón principal de la elección de Visual Studio Code es la familiaridad. He utilizado este editor desde que empecé mis estudios de grado y la personalización, extensiones, integración con Git y la facilidad de uso la hacen una herramienta tremendamente útil y versátil. Además, pueden integrarse terminales para ejecutar comandos y es multiplataforma, lo que permite trabajar en cualquier sistema operativo.\newline

\renewcommand{\icon}[1]{\includegraphics[height=18pt]{#1}}
\subsubsection*{SQLite Studio \protect\icon{./imagenes/sqlite_logo.png}}


SQLite Studio\footnote{\url{https://sqlitestudio.pl/}} es un administrador de bases de datos multiplataforma que permite navegar y editar archivos de bases de datos SQLite. Dispone de características como la creación de bases de datos, tablas, índices, vistas, procedimientos almacenados, funciones, etc. Además, permite ejecutar consultas SQL, importar y exportar datos, entre otras.\newline

La elección de SQLite Studio se debe a que es un administrador de bases de datos muy completo y fácil de usar. Permite gestionar varias bases de datos SQLite de forma sencilla y eficiente, lo que facilita el desarrollo y la depuración.\newline

\begin{figure}[H]
    \centering
    \includegraphics[width=1\textwidth]{./imagenes/SQLiteStudio.png}
    \caption{SQLite Studio.}
\end{figure}

\renewcommand{\icon}[1]{\includegraphics[height=18pt]{#1}}
\subsubsection*{Postman \protect\icon{./imagenes/postman_logo.png}}



Postman\footnote{\url{https://www.postman.com/}} es una plataforma para crear y utilizar API. Simplifica cada paso del ciclo de vida de una API y agiliza el desarrollo. Permite enviar solicitudes HTTP a un servidor y recibir respuestas. Dispone de características como la creación de \textit{colecciones de solicitudes}, la automatización de pruebas, la documentación de la API, entre otras. Es muy popular y cuenta con una gran cantidad de usuarios en todo el mundo.\newline

La elección de Postman se debe a una recomendación de mi tutor. Es una herramienta muy útil para probar API y asegurarse de que funcionan correctamente. Gracias a ella he podido probar los datos cruzados entre el backend y el frontend y depurar de manera más cómoda. Además, permite automatizar las pruebas y documentarlas.\newline

\renewcommand{\icon}[1]{\includegraphics[height=8pt]{#1}}
\subsubsection*{Textidote \protect\icon{./imagenes/textidote_logo.png}}



Textidote\footnote{\url{https://sylvainhalle.github.io/textidote/}} es un corrector ortográfico y gramatical de código abierto. Dispone de características como el resaltado de errores ortográficos y gramaticales, la sugerencia de sinónimos, la detección de frases redundantes, entre otras. Es muy útil para mejorar la calidad del código y evitar errores comunes. Fue diseñado para trabajar específicamente sobre Latex\newline

El motivo de la elección de esta herramienta fue para poder revisar el código de la documentación en LaTeX de este proyecto. Dado que el código contiene comandos especiales y palabras clave, no es posible aplicar correctores gramaticales comunes, a no ser que se extraiga el texto plano, pero Textidote está desarrollado para trabajar específicamente sobre Latex. Además, cuenta con una gran cantidad de usuarios en todo el mundo lo que hace más fácil encontrar soluciones a problemas comunes.\newline