\chapter{Análisis}

En esta sección se cubren los requisitos funcionales y no funcionales para el desarrollo del proyecto. (...)

\newpage
 
\section{Requisitos funcionales}

La calidad del software puede medirse en concordancia con los requisitos funcionales y de rendimiento que se establezcan, con las características intrínsecas que se esperan de cualquier software desarrollado de manera profesional. Los requisitos funcionales son aquellos que definen como debe comportarse el sistema, es decir, las funciones especificas que se esperan que cumplan con las necesidades finales del usuario \cite{Veloz_Segura_2022}.\newline

A continuación se describirán los requisitos funcionales, clasificados según su gestión principal, a fin de facilitar las comprensión clara de los funcionalidades que la plataforma debe incluir.

\subsection{Gestión de usuarios}

En este apartado se describen todas las funcionalidades relativas a los usuarios. Nótese que al referirnos al término ``usuario'' nos referimos a cualquier persona que haga uso de la plataforma, ya sea un administrador, un profesor o un alumno.
\newcounter{rfCounter}
\setcounter{rfCounter}{1}

\begin{table}[H]
    \centering
    \begin{tabular}{|p{4cm}|p{7cm}|}
    \hline
    \multicolumn{2}{|c|}{\textbf{RF\therfCounter\ - Registro de Usuario}} \\ \hline
    \textbf{Descripción} & Permite a un nuevo usuario registrarse en el sistema. \\ \hline
    \textbf{Datos de entrada} & Correo institucional. \\ \hline
    \textbf{Datos de salida} & Confirmación de registro. \\ \hline
    \end{tabular}
    \caption{RF\therfCounter\ - Registro de Usuario}
    \stepcounter{rfCounter}
\end{table}

\begin{table}[H]
    \centering
    \begin{tabular}{|p{4cm}|p{7cm}|}
    \hline
    \multicolumn{2}{|c|}{\textbf{RF\therfCounter\ - Inicio de Sesión}} \\ \hline
    \textbf{Descripción} & Permite a un usuario identificarse y acceder a sus datos. \\ \hline
    \textbf{Datos de entrada} & Credenciales del usuario. \\ \hline
    \textbf{Datos de salida} & Avatar del usuario en la pantalla de inicio y acceso a la opción de consultar sus horarios. \\ \hline
    \end{tabular}
    \caption{RF\therfCounter\ - Inicio de Sesión}
    \stepcounter{rfCounter}
\end{table}

\begin{table}[H]
    \centering
    \begin{tabular}{|p{4cm}|p{7cm}|}
    \hline
    \multicolumn{2}{|c|}{\textbf{RF\therfCounter\ - Baja de Usuario}} \\ \hline
    \textbf{Descripción} & Permite eliminar a un usuario del sistema y a sus datos asociados. \\ \hline
    \textbf{Datos de entrada} & Credenciales del usuario. \\ \hline
    \textbf{Datos de salida} & Confirmación de baja. \\ \hline
    \end{tabular}
    \caption{RF\therfCounter\ - Baja de Usuario}
    \stepcounter{rfCounter}
\end{table}

\subsection{Gestión de Calendarios}

En este apartado se describe la gestión de los calendarios semanales creados por el usuario con el fin de organizar su matrícula o su horario de clases. Nótese que al referirnos al término ``combinación'' estamos haciendo referencia a uno de los calendario semanales generados por el sistema con las asignaturas y grupos seleccionados por el usuario. 

\begin{table}[H]
    \centering
    \begin{tabular}{|p{4cm}|p{7cm}|}
    \hline
    \multicolumn{2}{|c|}{\textbf{RF\therfCounter\ - Seleccionar Grado}} \\ \hline
    \textbf{Descripción} & Permite al usuario seleccionar el grado de la UGR del que querrá organizar sus asignaturas. \\ \hline
    \textbf{Datos de entrada} & ?. \\ \hline
    \textbf{Datos de salida} & ?. \\ \hline
    \end{tabular}
    \caption{RF\therfCounter\ - Seleccionar Grado}
    \stepcounter{rfCounter}
\end{table}


\begin{table}[H]
    \centering
    \begin{tabular}{|p{4cm}|p{7cm}|}
    \hline
    \multicolumn{2}{|c|}{\textbf{RF\therfCounter\ - Seleccionar Curso}} \\ \hline
    \textbf{Descripción} & Permite al usuario seleccionar un curso académico. \\ \hline
    \textbf{Datos de entrada} & ?. \\ \hline
    \textbf{Datos de salida} & ?. \\ \hline
    \end{tabular}
    \caption{RF\therfCounter\ - Seleccionar Curso}
    \stepcounter{rfCounter}
\end{table}

\begin{table}[H]
    \centering
    \begin{tabular}{|p{4cm}|p{7cm}|}
    \hline
    \multicolumn{2}{|c|}{\textbf{RF\therfCounter\ - Seleccionar Cuatrimestre}} \\ \hline
    \textbf{Descripción} & Permite al usuario seleccionar un cuatrimestre del curso académico seleccionado. \\ \hline
    \textbf{Datos de entrada} & ?. \\ \hline
    \textbf{Datos de salida} & ?. \\ \hline
    \end{tabular}
    \caption{RF\therfCounter\ - Seleccionar Cuatrimestre}
    \stepcounter{rfCounter}
\end{table}

\begin{table}[H]
    \centering
    \begin{tabular}{|p{4cm}|p{7cm}|}
    \hline
    \multicolumn{2}{|c|}{\textbf{RF\therfCounter\ - Seleccionar Asignatura}} \\ \hline
    \textbf{Descripción} & Permite al usuario seleccionar una asignatura para incluirla en la planificación de su calendario semanal. \\ \hline
    \textbf{Datos de entrada} & ?. \\ \hline
    \textbf{Datos de salida} & ?. \\ \hline
    \end{tabular}
    \caption{RF\therfCounter\ - Seleccionar Asignatura}
    \stepcounter{rfCounter}
\end{table}

\begin{table}[H]
    \centering
    \begin{tabular}{|p{4cm}|p{7cm}|}
    \hline
    \multicolumn{2}{|c|}{\textbf{RF\therfCounter\ - Seleccionar Grupo}} \\ \hline
    \textbf{Descripción} & Permite al usuario seleccionar un grupo de teoría a planificar para la asignatura que ha previamente seleccionado. \\ \hline
    \textbf{Datos de entrada} & ?. \\ \hline
    \textbf{Datos de salida} & ?. \\ \hline
    \end{tabular}
    \caption{RF\therfCounter\ - Seleccionar Grupo}
    \stepcounter{rfCounter}
\end{table}

\begin{table}[H]
    \centering
    \begin{tabular}{|p{4cm}|p{7cm}|}
    \hline
    \multicolumn{2}{|c|}{\textbf{RF\therfCounter\ - Seleccionar Combinación}} \\ \hline
    \textbf{Descripción} & Permite al usuario seleccionar una combinación de las generadas por el sistema con su selección de asignaturas y grupos. \\ \hline
    \textbf{Datos de entrada} & ?. \\ \hline
    \textbf{Datos de salida} & ?. \\ \hline
    \end{tabular}
    \caption{RF\therfCounter\ - Seleccionar Calendario}
    \stepcounter{rfCounter}
\end{table}

\begin{table}[H]
    \centering
    \begin{tabular}{|p{4cm}|p{7cm}|}
    \hline
    \multicolumn{2}{|c|}{\textbf{RF\therfCounter\ - Descargar Combinación}} \\ \hline
    \textbf{Descripción} & Permite al usuario descargar la combinación deseada en formato PDF. \\ \hline
    \textbf{Datos de entrada} & ?. \\ \hline
    \textbf{Datos de salida} & Calendario de la combinación seleccionada en formato PDF. \\ \hline
    \end{tabular}
    \caption{RF\therfCounter\ - Descargar Combinación}
    \stepcounter{rfCounter}
\end{table}

\begin{table}[H]
    \centering
    \begin{tabular}{|p{4cm}|p{7cm}|}
    \hline
    \multicolumn{2}{|c|}{\textbf{RF\therfCounter\ - Guardar Calendario}} \\ \hline
    \textbf{Descripción} & Permite al usuario guardar la combinación seleccionada en el sistema para consultarla posteriormente. \\ \hline
    \textbf{Datos de entrada} & ?. \\ \hline
    \textbf{Datos de salida} & ?. \\ \hline
    \end{tabular}
    \caption{RF\therfCounter\ - Guardar Calendario}
    \stepcounter{rfCounter}
\end{table}

\begin{table}[H]
    \centering
    \begin{tabular}{|p{4cm}|p{7cm}|}
    \hline
    \multicolumn{2}{|c|}{\textbf{RF\therfCounter\ - Consultar Calendario}} \\ \hline
    \textbf{Descripción} & Permite al usuario consultar sus calendarios finales. \\ \hline
    \textbf{Datos de entrada} & ?. \\ \hline
    \textbf{Datos de salida} & ?. \\ \hline
    \end{tabular}
    \caption{RF\therfCounter\ - Consultar Calendario}
    \stepcounter{rfCounter}
\end{table}

\section{Requisitos no funcionales}

Los requisitos no funcionales son aquellos que no se refieren específicamente a la funcionalidad de un sistema. Imponen restricciones sobre el producto que se está desarrollando y el proceso de desarrollo, y especifican restricciones externas que debe cumplir el producto \cite{nonFR}.

\newcounter{nrfCounter}
\setcounter{nrfCounter}{1}

\begin{table}[H]
    \centering
    \begin{tabular}{|p{4cm}|p{7cm}|}
    \hline
    \multicolumn{2}{|c|}{\textbf{RNF\thenrfCounter\ - Usabilidad}} \\ \hline
    \textbf{Descripción} & La plataforma debe ser fácil e intuitiva para los usuarios. \\ \hline
    \textbf{Criterios} & Interfaz de usuario sencilla y autoexplicativa. \\ \hline
    \end{tabular}
    \caption{RNF\thenrfCounter\ - Usabilidad}
    \stepcounter{nrfCounter}
\end{table}

\begin{table}[H]
    \centering
    \begin{tabular}{|p{4cm}|p{7cm}|}
    \hline
    \multicolumn{2}{|c|}{\textbf{RNF\thenrfCounter\ - Rendimiento}} \\ \hline
    \textbf{Descripción} & La plataforma debe responder con rapidez a las consultas del usuario. \\ \hline
    \textbf{Criterios} & Tiempos de respuesta rápidos para las peticiones. \\ \hline
    \end{tabular}
    \caption{RNF\thenrfCounter\ - Rendimiento}
    \stepcounter{nrfCounter}
\end{table}

\begin{table}[H]
    \centering
    \begin{tabular}{|p{4cm}|p{7cm}|}
    \hline
    \multicolumn{2}{|c|}{\textbf{RNF\thenrfCounter\ - Eficiencia}} \\ \hline
    \textbf{Descripción} & La plataforma debe reducir el tiempo que el usuario tardaría en hacer su calendario semanal manualmente. \\ \hline
    \textbf{Criterios} & Proceso rápido y automático de generación del calendario. \\ \hline
    \end{tabular}
    \caption{RNF\thenrfCounter\ - Eficiencia}
    \stepcounter{nrfCounter}
\end{table}

\begin{table}[H]
    \centering
    \begin{tabular}{|p{4cm}|p{7cm}|}
    \hline
    \multicolumn{2}{|c|}{\textbf{RNF\thenrfCounter\ - Compatibilidad}} \\ \hline
    \textbf{Descripción} & La plataforma funcionará correctamente independientemente del navegador o dispositivo usado. \\ \hline
    \textbf{Criterios} & React permite un diseño responsive y React Native permite desplegar archivos APK para Android y archivos para iOS  . \\ \hline
    \end{tabular}
    \caption{RNF\thenrfCounter\ - Compatibilidad}
    \stepcounter{nrfCounter}
\end{table}

\begin{table}[H]
    \centering
    \begin{tabular}{|p{4cm}|p{7cm}|}
    \hline
    \multicolumn{2}{|c|}{\textbf{RNF\thenrfCounter\ - Mantenibilidad}} \\ \hline
    \textbf{Descripción} & La plataforma facilitará las actualizaciones y el desarrollo de su contenido. \\ \hline
    \textbf{Criterios} & La documentación y el código de la plataforma se desarrollarán de forma clara para facilitar su entendimiento y posterior actualización. \\ \hline
    \end{tabular}
    \caption{RNF\thenrfCounter\ - Mantenibilidad}
    \stepcounter{nrfCounter}
\end{table}

\section{Modelo de Caso de Uso}

Los casos de uso son una secuencia de eventos que en conjunto, conducen a un sistema haciendo algo útil. Describen las diferentes formas en que los usuarios interaccionan con el mismo y como éste les responde. Son una herramienta fundamental para la comprensión de los requisitos funcionales y ayudan a entender las funcionalidades esperadas del sistema y las partes implicadas \cite{bittner2003use}.

\subsection{Actores del Sistema}

La interacción con la plataforma se basa en dos tipos de actores: los usuarios genéricos y el administrador(?). El acceso no está restringido a usuarios registrados, pero para poder guardar sus calendarios y consultarlos posteriormente, los usuarios deberán registrarse en la plataforma.

\begin{enumerate}
    \item \textbf{Usuario genérico}: Son los usuarios que acceden a la plataforma para generar sus calendarios semanales y es principalmente a quien va dirigida la plataforma. Pueden seleccionar el grado, curso, cuatrimestre, asignaturas y grupos que desean incluir en su calendario semanal. Consultar los calendarios generados, descargarlos en formato PDF y guardarlos en la plataforma para consultarlos posteriormente.
    \item \textbf{Administrador}: (?)
\end{enumerate}

\subsection{Escenarios de Casos de Uso}

\newcounter{ccCounter}
\setcounter{ccCounter}{1}


\begin{table}[H]
    \centering
    \begin{tabular}{|p{4cm}|p{7cm}|}
    \hline
    \multicolumn{2}{|c|}{\textbf{CU\theccCounter\ - Registrar Usuario}} \\ \hline
    \textbf{Actores} & Usuario Genérico. \\ \hline
    \textbf{Precondiciones} & El actor no está registrado en el sistema. \\ \hline
    \textbf{Postcondiciones} & El actor está registrado en el sistema. \\ \hline
    \textbf{Flujo principal} & \begin{minipage}[t]{\linewidth}
        \vspace{1pt} % Ajusta esto para reducir el espacio superior
        \begin{enumerate}
            \setlength{\itemsep}{0pt}
            \setlength{\parskip}{0pt}
            \setlength{\parsep}{0pt}
            \item El actor accede a la plataforma.
            \item El actor ingresa su correo institucional.
            \item El sistema envía un correo de confirmación al correo institucional del actor.
            \item El actor confirma su registro en la plataforma.
            \item El sistema registra al actor en la plataforma.
        \end{enumerate}
        \vspace{1pt} % Ajusta esto para reducir el espacio inferior
    \end{minipage} \\ \hline  
    \end{tabular}
    \caption{CU\theccCounter\ - Registrar Usuario}
    \stepcounter{ccCounter}
\end{table}

\begin{table}[H]
    \centering
    \begin{tabular}{|p{4cm}|p{7cm}|}
    \hline
    \multicolumn{2}{|c|}{\textbf{CU\theccCounter\ - Autenticar Usuario}} \\ \hline
    \textbf{Actores} & Usuario Genérico. \\ \hline
    \textbf{Precondiciones} & El actor está registrado en el sistema. \\ \hline
    \textbf{Postcondiciones} & El actor está autenticado en el sistema. \\ \hline
    \textbf{Flujo principal} & \begin{minipage}[t]{\linewidth}
        \vspace{1pt} % Ajusta esto para reducir el espacio superior
        \begin{enumerate}
            \setlength{\itemsep}{0pt}
            \setlength{\parskip}{0pt}
            \setlength{\parsep}{0pt}
            \item El actor accede a la plataforma.
            \item El actor ingresa sus credenciales.
            \item El sistema valida sus credenciales.
            \item El sistema muestra el avatar del actor en la pantalla de inicio y le da acceso a la opción de consultar sus horarios.
        \end{enumerate}
        \vspace{1pt} % Ajusta esto para reducir el espacio inferior
    \end{minipage} \\ \hline  
    \end{tabular}
    \caption{CU\theccCounter\ - Autenticar Usuario}
    \stepcounter{ccCounter}
\end{table}


\begin{table}[H]
    \centering
    \begin{tabular}{|p{4cm}|p{7cm}|}
    \hline
    \multicolumn{2}{|c|}{\textbf{CU\theccCounter\ - Eliminar Usuario}} \\ \hline
    \textbf{Actores} & Usuario Genérico. \\ \hline
    \textbf{Precondiciones} & El actor está registrado en el sistema. \\ \hline
    \textbf{Postcondiciones} & El actor no está registrado en el sistema. \\ \hline
    \textbf{Flujo principal} & \begin{minipage}[t]{\linewidth}
        \vspace{1pt} % Ajusta esto para reducir el espacio superior
        \begin{enumerate}
            \setlength{\itemsep}{0pt}
            \setlength{\parskip}{0pt}
            \setlength{\parsep}{0pt}
            \item El actor accede a la plataforma.
            \item El actor selecciona la opción de eliminar su cuenta.
            \item El sistema solicita confirmación al actor.
            \item El actor confirma la eliminación de su cuenta.
            \item El sistema elimina al actor de la plataforma y sus datos asociados.
        \end{enumerate}
        \vspace{1pt} % Ajusta esto para reducir el espacio inferior
    \end{minipage} \\ \hline  
    \end{tabular}
    \caption{CU\theccCounter\ - Eliminar Usuario}
    \stepcounter{ccCounter}
\end{table}

% Pongo cada acción como un caso o todo el proceso junto??

\begin{table}[H]
    \centering
    \begin{tabular}{|p{4cm}|p{7cm}|}
    \hline
    \multicolumn{2}{|c|}{\textbf{CU\theccCounter\ - Seleccionar Grado}} \\ \hline
    \textbf{Actores} & Usuario Genérico. \\ \hline
    \textbf{Precondiciones} & El actor accede a la plataforma. \\ \hline
    \textbf{Postcondiciones} & Se muestran las asignaturas del grado escogido. \\ \hline
    \textbf{Flujo principal} & \begin{minipage}[t]{\linewidth}
        \vspace{1pt} % Ajusta esto para reducir el espacio superior
        \begin{enumerate}
            \setlength{\itemsep}{0pt}
            \setlength{\parskip}{0pt}
            \setlength{\parsep}{0pt}
            \item El actor accede a la plataforma.
            \item El selecciona el grado de interés.
            \item El sistema envía un correo de confirmación al correo institucional del actor.
            \item El actor confirma su registro en la plataforma.
            \item El sistema registra al actor en la plataforma.
        \end{enumerate}
        \vspace{1pt} % Ajusta esto para reducir el espacio inferior
    \end{minipage} \\ \hline  
    \end{tabular}
    \caption{CU\theccCounter\ - Seleccionar Grado}
    \stepcounter{ccCounter}
\end{table}


\begin{table}[H]
    \centering
    \begin{tabular}{|p{4cm}|p{7cm}|}
    \hline
    \multicolumn{2}{|c|}{\textbf{CU\theccCounter\ - Consultar Horarios}} \\ \hline
    \textbf{Actores} & Usuario Genérico. \\ \hline
    \textbf{Precondiciones} & El actor está registrado, autenticado en el sistema y ha guardado algún calendario. \\ \hline
    \textbf{Postcondiciones} & Se muestran los calendarios del actor. \\ \hline
    \textbf{Flujo principal} & \begin{minipage}[t]{\linewidth}
        \vspace{1pt} % Ajusta esto para reducir el espacio superior
        \begin{enumerate}
            \setlength{\itemsep}{0pt}
            \setlength{\parskip}{0pt}
            \setlength{\parsep}{0pt}
            \item El actor accede a la plataforma.
            \item El actor ingresa sus crendenciales.
            \item El sistema autentifica al usuario.
            \item El actor selecciona la opción ``Consultar Mis Horarios''.
            \item El sistema muestra sus calendarios guardados.
        \end{enumerate}
        \vspace{1pt} % Ajusta esto para reducir el espacio inferior
    \end{minipage} \\ \hline  
    \end{tabular}
    \caption{CU\theccCounter\ - Consultar Horarios}
    \stepcounter{ccCounter}
\end{table}
