\chapter{Planificación}
\section{Presupuesto}

En esta sección se examinan y detallan los costos del proyecto. Se considerará el perfil de un ingeniero informático Junior, tomando en cuenta que el salario anual oscila entre los 20.000 € - 28.000 € \footnote{\url{https://www.glassdoor.es/Sueldos/junior-software-engineer-sueldo-SRCH_KO0,24.htm}}. Teniendo en cuenta que la implementación ha supuesto un aproximado de 200 horas. Partiendo de esta información, se puede calcular el coste total del salario del desarrollador, el cual se detalla a continuación:

\begin{equation}
    \textbf{Salario medio Ingeniero Junior} =  \frac {\text{24.000 €/Año} }{ \text{12 meses}} = \text{2.000 €/Mes}
\end{equation}

El salario por hora se calcula dividiendo el salario mensual entre 160 horas, que es el promedio de horas laborales al mes:

\begin{equation}
    \textbf{Salario por Hora} = \frac {\text{2.000 €/Mes}}{160 \text{ Horas}} = \text{12,50 €/Hora}
\end{equation}

Teniendo en cuenta las 200 horas de trabajo, se puede calcular el coste total del salario del desarrollador:

\begin{equation}
    \textbf{Coste Salario} = \text{200 Horas} \times \text{12,50 €/Hora} = \text{2.500 €}
\end{equation}

Para el diseño, se tendrá en cuenta el perfil de un diseñador gráfico, cuyo salario anual oscila entre los 17.000 € - 22.000 €. Teniendo en cuenta que el diseño ha supuesto un aproximado de 30 horas. Partiendo de esta información, se puede calcular el coste total del salario del diseñador, el cual se detalla a continuación:

\begin{equation}
    \textbf{Salario medio Diseñador Gráfico} =  \frac {\text{19.500 €/Año} }{ \text{12 meses}} = \text{1.625 €/Mes}
\end{equation}

El salario por hora se calcula dividiendo el salario mensual entre 160 horas, que es el promedio de horas laborales al mes:

\begin{equation}
    \textbf{Salario por Hora} = \frac {\text{1.625 €/Mes}}{160 \text{ Horas}} = \text{10,16 €/Hora}
\end{equation}

Teniendo en cuenta las 30 horas de trabajo, se puede calcular el coste total del salario del diseñador:

\begin{equation}
    \textbf{Coste Salario} = \text{30 Horas} \times \text{10,16 €/Hora} = \text{304,80 €}
\end{equation}

En cuanto a las licencias necesarias para este proyecto, se han utilizado exclusivamente tecnologías de código abierto con el propósito de reducir costos. Como resultado de esta decisión, no ha habido ningún gasto en licencias. Para calcular el coste del hardware mencionado en la sección de materiales, es esencial tener en cuenta la vida útil estimada de los dispositivos, como un portátil, que se estima en 5 años \footnote{\url{https://www.minitool.com/es/respaldar-datos/cuanto-dura-un-ordenador-portatil.html}}.

\begin{equation}
    \textbf{Coste Anual} = \frac {\text{Coste Materiales}}{\text{5 años}} = \text{169,38 €/año}
\end{equation}

\begin{equation}
    \textbf{Materiales x 8 Meses} = \left(\frac {\text{Coste Materiales}}{\text{5 años}}\right) \times 8 \text{ Meses} = \text{1.355,04 €}
\end{equation}

Sumando los costes de salario y materiales, se obtiene el coste total del proyecto (Figura \ref{tab:coste_total}).

\begin{table}[H]
    \centering
    \begin{tabular}{|c|c|ll}
    \cline{1-2}
    \multicolumn{1}{|l|}{Concepto} & \multicolumn{1}{l|}{Coste (€)} &  &  \\ \cline{1-2}
    Salario Desarrollador           & 2.500,00                        &  &  \\ \cline{1-2}
    Salario Diseñador               & 304,80                         &  &  \\ \cline{1-2}
    Materiales                      & 1.355,04                       &  &  \\ \cline{1-2}
    \textbf{Total}                   & \textbf{4.159,84}               &  &  \\ \cline{1-2}
    \end{tabular}
    \caption{Coste Total del Proyecto}
    \label{tab:coste_total}
\end{table}
