\chapter{Planificación}

Para abordar el problema discutido en la sección anterior y superar las limitaciones que presentan las soluciones actuales, se desarrollará \textbf{GAC} (Gestor Académico de Calendarios), una plataforma web innovadora diseñada para los estudiantes de la Universidad de Granada. Esta plataforma facilitará la gestión de sus horarios académicos de manera intuitiva, sencilla y eficiente. GAC no solo integrará las funcionalidades básicas de las soluciones ya existentes, sino que también las potenciará con una serie de características adicionales que harán la experiencia del usuario mucho más completa y personalizada. Dichas características incluyen:

\begin{enumerate} 
    \item \textbf{Centralización de la información:} GAC reunirá en un único lugar toda la información académica relevante para los estudiantes, como asignaturas, grupos y horarios. Esta centralización permitirá acceder de forma rápida y sencilla a todos los datos necesarios, evitando la dispersión de la información en diferentes plataformas o documentos.

    \item \textbf{Gestión de horarios optimizada:} La plataforma ofrecerá una gestión automática de horarios, buscando evitar solapamientos entre clases y garantizando la distribución más óptima de las asignaturas a lo largo de la semana. Los estudiantes podrán personalizar sus horarios según sus preferencias y necesidades, lo que les permitirá planificar su tiempo de manera más efectiva.

    \item \textbf{Exclusividad en el enfoque:} GAC se destacará por su enfoque exclusivo en la gestión de calendarios académicos, ofreciendo una solución especializada y adaptada a las necesidades específicas de los estudiantes universitarios. A diferencia de otras herramientas genéricas, GAC se centrará en proporcionar una experiencia más completa y adaptada a los usuarios, mejorando significativamente las funcionalidades ya disponibles y añadiendo nuevas características para un manejo más eficiente y personalizado. 
\end{enumerate}

Con estas mejoras, GAC se posicionará como una herramienta indispensable para los estudiantes de la Universidad de Granada, facilitando su organización académica y contribuyendo a un mejor aprovechamiento de su tiempo y recursos.

\section{Metodología de Desarrollo Ágil}

La ingeniería de software es una disciplina que ha evolucionado significativamente para adaptarse a los avances tecnológicos y a las necesidades cambiantes de las empresas modernas. Esto se ha logrado mediante la creación de métodos efectivos que guían el proceso hasta alcanzar el producto final de software. Uno de estos métodos exitosos es el desarrollo ágil de software. Este enfoque ágil es más ligero y fue diseñado para superar las limitaciones de los métodos de desarrollo más tradicionales y complejos, reduciendo así tanto el costo como la carga de trabajo. Al mismo tiempo, ofrece una gran flexibilidad para incorporar cambios en los requisitos en cualquier fase del proyecto, lo que se logra a través de una gestión y coordinación de tareas basada en un conjunto específico de valores y principios fundamentales \cite{AlSaqqa2020AgileSD}.\newline

Dichos valores y principios se encuentran recogidos en el Manifiesto Ágil \cite{beck2001agile}. Algunas de esas premisas son:

\begin{itemize}
    \item \textbf{Flexibilidad y adaptabilidad:} el desarrollo ágil se basa en la capacidad de adaptarse a los cambios y responder a ellos de manera efectiva.
    \item \textbf{Desarrollo iterativo e incremental:} el desarrollo ágil se basa en la creación de versiones del producto en ciclos cortos y regulares, lo que permite obtener retroalimentación temprana y realizar ajustes en función de ella.
    \item \textbf{Entregas continuas:} el desarrollo ágil se basa en la idea de que es mejor entregar un producto funcional en partes pequeñas y regulares que esperar a tener un producto completo.
    \item \textbf{Simplicidad:} el desarrollo ágil se basa en la idea de que la simplicidad es esencial para el éxito del proyecto. Se busca minimizar la complejidad y centrarse en lo esencial.
\end{itemize}

\subsection{Ejemplos de Metodologías Ágiles}

\begin{enumerate}
    \item \textbf{Kanban:} Se trata de un método visual para la gestión de proyectos que se basa en la utilización de tableros y tarjetas para representar las tareas y su estado. Kanban es una metodología ágil que se centra en la mejora continua y en la optimización del flujo de trabajo. En un tablero Kanban (Figura \ref{fig:kanban}), las tareas se dividen en columnas que representan diferentes estados del proceso, como "pendiente", "en progreso" y "completado". Los miembros del equipo pueden mover las tarjetas de una columna a otra a medida que avanzan en su trabajo, lo que permite visualizar de forma clara y sencilla el estado de las tareas y identificar posibles cuellos de botella o retrasos.
    \begin{figure}[H]
        \centering
        \includegraphics[width=0.6\textwidth]{imagenes/kanban.png}
        \caption{Ejemplo de tablero Kanban \cite{kirovska2015usage}.}
        \label{fig:kanban}
    \end{figure}
    \item \textbf{Scrum:} Es un marco de trabajo ágil que se centra en la colaboración, la transparencia y la adaptabilidad. En Scrum, los proyectos se dividen en iteraciones cortas llamadas ``sprints'' (Figura \ref{fig:scrum}), que suelen tener una duración de 2 a 4 semanas. Durante cada sprint, el equipo se compromete a completar un conjunto de tareas específicas y al final del sprint se realiza una revisión para evaluar los resultados y planificar el siguiente sprint. Scrum se basa en roles definidos, como el ``Product Owner'', el ``Scrum Master'' y el ``Equipo de Desarrollo'', que trabajan juntos para lograr los objetivos del proyecto.
    \begin{figure}[H]
        \centering
        \includegraphics[width=0.6\textwidth]{imagenes/scrum.png}
        \caption{Ejemplo de tablero Scrum \cite{karabiyik2020understanding}.}
        \label{fig:scrum}
    \end{figure}
\end{enumerate}



\section{Temporización}

\section{Recursos y costes}

En este apartado se discuten los recursos y materiales empleados en el desarrollo del proyecto, así como los costes asociados a los mismos.

\subsection{Recursos humanos}

Para el desarrollo de la plataforma \textbf{GAC}, se ha contado con un único desarrollador, el autor de este documento. Además, cabe destacar la labor de mis tutores, quienes han supervisado y guiado el desarrollo del mismo.

\subsection{Materiales}

Para el desarrollo de la plataforma se ha utilizado un portátil personal y un segundo monitor, el cual ha sido de gran utilidad para la realización de tareas de desarrollo y diseño, al permitir visualizar de forma simultánea varias ventanas y herramientas. Ambos dispositivos son propiedad del autor.

% Please add the following required packages to your document preamble:
% \usepackage{multirow}
\begin{table}[H]
    \begin{tabular}{|c|c|c|ll}
    \cline{1-3}
    \multicolumn{1}{|l|}{Concepto}                                                                              & \multicolumn{1}{l|}{Coste Unitario (€)} & \multicolumn{1}{l|}{Coste Materiales (€)} &  &  \\ \cline{1-3}
    \begin{tabular}[c]{@{}c@{}}Acer Aspire 3 A315-59-504M -\\  Intel® Core™ i5-1235U -\\  16GB RAM\end{tabular} & 749 €                                   & \multirow{2}{*}{846,90€}                  &  &  \\ \cline{1-2}
    \begin{tabular}[c]{@{}c@{}}Monitor -\\ Asus VZ239HE 23" Full HD IPS\end{tabular}                            & 97,90€                                  &                                           &  &  \\ \cline{1-3}
    \end{tabular}
    \caption{Coste de Materiales}
\end{table}

\section{Presupuesto}

En esta sección se examinan y detallan los costos del proyecto empleando datos obtenidos de las bases de cotización para contingencias comunes correspondientes al año 2024, según la Seguridad Social de España \cite{seg-social}. Se considerará el perfil de un ingeniero informático junior, tomando en cuenta que las bases de cotización varían entre un mínimo de 1.847,40 € y un máximo de 4.720,50 €. Para el cálculo, se utilizará la base mínima de cotización y se asumirá que la duración del proyecto será de 6 meses.\newline

Partiendo de esta información, se puede calcular el coste total del salario del desarrollador, el cual se detalla a continuación:

\begin{equation}
    \textbf{Salario Ingeniero Junior} = \text{1.847,40 €/mes} \times 6 \text{ meses} = \text{11.084,40 €}
\end{equation}

En cuanto a las licencias necesarias para este proyecto, se han utilizado exclusivamente tecnologías de código abierto con el propósito de reducir costos. Como resultado de esta decisión, no ha habido ningún gasto en licencias. Para calcular el coste del hardware mencionado en la sección de materiales, es esencial tener en cuenta la vida útil estimada de los dispositivos, como un portátil, que se estima en 5 años \footnote{\url{https://www.minitool.com/es/respaldar-datos/cuanto-dura-un-ordenador-portatil.html}}.

\begin{equation}
    \textbf{Coste Anual} = \frac {\text{Coste Materiales}}{\text{5 años}} = \text{169,38 €/año}
\end{equation}

\begin{equation}
    \textbf{Materiales x 6 Meses} = \left(\frac {\text{Coste Materiales}}{\text{5 años}}\right) \times 6 \text{ Meses} = \text{1.016,28 €}
\end{equation}

Sumando los costes de salario y materiales, se obtiene el coste total del proyecto:

\begin{table}[H]
    \begin{tabular}{|c|c|c|l|l}
    \cline{1-4}
    Concepto         & Mensual    & 6 Meses     & \multicolumn{1}{c|}{Materiales + Desarrollo} &  \\ \cline{1-4}
    Ingeniero Junior & 1.847,90 € & 11.084,40 € &                                              &  \\ \cline{1-4}
    \end{tabular}
\end{table}