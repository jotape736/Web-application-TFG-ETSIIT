\chapter{Introducción}


\section{Contexto y Motivación}
El uso de las TICS (Tecnologías de la Información y la Comunicación) se encuentra patente en entornos académicos y escolares. Estas tecnologías, a través de numerosas herramientas, mejoran significativamente los procesos de aprendizaje, la interacción entre profesor y estudiantado y, en definitiva, hacen mejor la experiencia educativa \cite{Flores-Alarcia2012,Paladines,TICenLaEducacion}.\newline

Dentro de estas herramientas, algunas se emplean para gestionar de forma eficiente, y en algunos casos de manera automática, tareas propias de personal administrativo, como el alta de nuevos estudiantes, registro de asignaturas, publicación de notas, etc. No obstante, a pesar el avance tecnológico, algunas tareas siguen realizándose manualmente, por la complicación que supone su automatización o bien por la falta de herramientas. Una tarea que aún se sigue realizando mayormente de manera manual es la generación de calendarios académicos personalizados.\newline

Entre las 27 facultades y escuelas \footnote[1]{\url{https://www.ugr.es/universidad/organizacion/facultades-escuelas}} que componen la universidad de Granada hay una oferta académica de 62 títulos de grado \footnote[2]{\url{https://www.ugr.es/destacado/oferta-academica-grados-masteres-y-doctorados}}. Los créditos totales que deben cursarse en la titulación serán de 240 ECTS \footnote[3]{\url{https://grados.ugr.es/documentacion/pages/infoacademica/estudios}}, distribuidos en 4 años. Los 60 pertenecientes a la formación básica se reparten entre 21 materias obligatorias (además del Trabajo Fin de Grado y las Prácticas Externas) y 13 optativas.\newline

Dichas asignaturas se imparten en grupos de teoría y sus correspondientes subgrupos de prácticas que se distribuyen en diferentes horarios a lo largo de la semana. Debido al modelo educativo vigente, el estudiante puede matricularse en varias asignaturas de distintos cursos simultáneamente.\newline

Todos los años el estudiante debe encargarse de consultar la información de las asignaturas en las que desea matricularse, así como los horarios de las clases y grupos de prácticas. A partir de esta información, el estudiante debe generar un calendario semanal que le permita visualizar de forma clara y concisa la distribución de las asignaturas. Esta tarea, aunque sencilla, puede resultar tediosa y propensa a errores, ya que el estudiante debe tener en cuenta la disponibilidad, los solapamientos de horarios, etc.\newline

Una vez pasado el proceso de matriculación, secretaría se encarga de hacer una distribución equilibrada de los estudiantes en los grupos de teoría y prácticas que les han sido asignados. Dicha asignación depende de varios factores tales como la demanda de plazas, la nota media del expediente académico, etc. Por otro lado, se priorizan las solicitudes de los estudiantes que seleccionan grupos completos para sus asignaturas sobre los que eligen grupos sueltos. Dicho proceso responde, principalmente, a la necesidad de evitar solapamientos, pero no se ajusta en todos los casos a las preferencias del estudiante.\newline

En el caso de alumnos con materias pendientes o estudiantes provenientes de ciclos formativos, el proceso no es tan sencillo como matricularse en todas las asignaturas del año que estén cursando.
Si un estudiante de ciclo formativo tiene convalidadas gran parte de las asignaturas de un año, o tiene pendientes asignaturas de cursos anteriores, es presumible que querrá matricularse en asignaturas de cursos superiores o inferiores, a fin de no perder tiempo y ni dinero (por motivo de becas, por ejemplo). En este contexto, la tarea de organizar un calendario ya no es tan sencilla, más aún si se tiene que elegir un grupo concreto para convalidar unas prácticas aprobadas de años anteriores por ejemplo.\newline

En este proyecto se pretende abordar la tarea de facilitar a los estudiantes de la Universidad de Granada la generación de un calendario académico semanal a partir de las asignaturas y grupos en los que deseen matricularse. Reduciendo así la carga de trabajo y el tiempo invertido en dicha tarea.


\section{Objetivos}
El objetivo fundamental del presente proyecto es la \textit{creación de una plataforma (\textbf{GAC}: Gestor Académico de Calendarios) que permita a los estudiantes de la Universidad de Granada, generar un calendario académico semanal a partir de las asignaturas y grupos de prácticas en los que deseen matricularse.}\newline

Para la consecución de este objetivo principal, se han establecido los siguientes objetivos específicos:
\begin{enumerate}
    \item Revisión de soluciones existentes para la generación automática de calendarios académicos. Tanto comerciales como proyectos de investigación.
    \item Estudio de técnicas de timetabling.
    \item Estudio del uso de IA generativa para la solución del problema de timetabling.
    \item Estudio de nuevas tecnologías y paradigmas de computación, tales como tecnologías web, arquitecturas orientadas a servicios, etc.
\end{enumerate}