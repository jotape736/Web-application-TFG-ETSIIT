\thispagestyle{empty}

\begin{center}
	\newpage
{\large\bfseries Plataforma para la gestión de calendarios académicos basada en servicios web}\\
\end{center}
\begin{center}
José Pablo Márquez Megías\\
\end{center}

%\vspace{0.7cm}

\vspace{0.5cm}
\noindent\textbf{Palabras clave}: \textit{software libre}, \textit{timetabling}, \textit{application programming interfaces}, \textit{web scraping}, \textit{servicios web}, \textit{arquitectura orientada a servicios}
\vspace{0.7cm}

\noindent\textbf{Resumen}\\

Cada año los estudiantes deben matricularse en una serie de asignaturas, organizadas en teoría y prácticas con horarios distribuidos a lo largo de la semana. Este proceso de matrícula, aunque fundamental, puede ser tedioso y propenso a errores, especialmente para aquellos con asignaturas pendientes de otros cursos o provenientes de ciclos formativos que deben combinar asignaturas de distintos cursos y grupos en un mismo cuatrimestre.\newline

El presente proyecto describe el desarrollo de una plataforma para la gestión de calendarios académicos, que permite a los estudiantes de la Universidad de Granada generar de forma automática un calendario semanal a partir de las asignaturas y grupos en los que deseen matricularse. La plataforma, denominada \textbf{GAC} (Gestor Académico de Calendarios), centraliza y mantiene actualizada toda la información relevante sobre la planificación docente, además de aportar una serie de herramientas que permiten a los alumnos la planificación automática de su selección de matrícula reduciendo significativamente el tiempo que les tomaría hacerlo de forma manual y sin errores.\newline

\cleardoublepage

\begin{center}
	{\large\bfseries Platform for the management of academic calendars based on web services}\\
\end{center}
\begin{center}
	José Pablo Márquez Megías\\
\end{center}
\vspace{0.5cm}
\noindent\textbf{Keywords}: \textit{open source}, \textit{timetabling}, \textit{web services}, \textit{application programming interfaces}, \textit{web scraping}, \textit{service-oriented architecture}
\vspace{0.7cm}

\noindent\textbf{Abstract}\\

Each year students must enroll in a series of subjects, organized in theory and practice with schedules distributed throughout the week. This enrollment process, although fundamental, can be tedious and error-prone, especially for those students with pending subjects from other courses or those coming from training cycles who must combine subjects from different courses and groups in the same term.\newline

This project describes the development of a platform for the management of academic calendars, which allows students at the University of Granada to automatically generate a weekly calendar based on the subjects and groups in which they wish to enroll. The platform, called GAC (Gestor Académico de Calendarios), centralizes and keeps updated all the relevant information about teaching planning, as well as providing a series of tools that allow students to automatically plan their enrollment selection, significantly reducing the time it would take them to do it manually and without errors.

\cleardoublepage

\thispagestyle{empty}

\noindent\rule[-1ex]{\textwidth}{2pt}\\[4.5ex]

D. \textbf{Ángel Ruíz Zafra} y Dª. \textbf{Kawtar Benghazi Akhlaki Sekkate}, Profesores del departamento de Lenguajes y Sistemas Informáticos.
\vspace{0.5cm}

\textbf{Informamos:}

\vspace{0.5cm}

Que el presente trabajo, titulado \textit{\textbf{Servicio web para la gestión de calendarios académicos}},
ha sido realizado bajo nuestra supervisión por \textbf{José Pablo Márquez Megías}, y autorizamos la defensa de dicho trabajo ante el tribunal
que corresponda.

\vspace{0.5cm}

Y para que conste, expiden y firman el presente informe en Granada a septiembre de 2024.

\vspace{1cm}

\textbf{El/la director(a)/es: }

\vspace{5cm}

\noindent \textbf{Ángel Ruíz Zafra} y \textbf{Kawtar Benghazi Akhlaki Sekkate}

\chapter*{Agradecimientos}

A mi familia, porque sin ellos no habría llegado hasta aquí. A mis amigos, por su apoyo incondicional. A mis tutores, por su paciencia y dedicación.
