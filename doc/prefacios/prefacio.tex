\thispagestyle{empty}

\begin{center}
{\large\bfseries Servicio web para la gestión de calendarios académicos}\\
\end{center}
\begin{center}
José Pablo Márquez Megías\\
\end{center}

%\vspace{0.7cm}

\vspace{0.5cm}
\noindent\textbf{Palabras clave}: \textit{software libre}, \textit{timetabling}, \textit{API}, \textit{web scraping}, \textit{SOA}
\vspace{0.7cm}

\noindent\textbf{Resumen}\\

Cada año los estudiantes deben matricularse en una serie de asignaturas, organizadas en teoría y prácticas con horarios distribuidos a lo largo de la semana. Este proceso de matrícula, aunque fundamental, puede ser tedioso y propenso a errores, especialmente para aquellos con asignaturas pendientes de otros cursos o provenientes de ciclos formativos que deben combinar asignaturas de distintos cursos y grupos en un mismo cuatrimestre.\newline

El presente proyecto describe el desarrollo de una plataforma para la gestión de calendarios académicos, que permite a los estudiantes de la Universidad de Granada generar de forma automática un calendario semanal a partir de las asignaturas y grupos en los que deseen matricularse. La plataforma, denominada \textbf{GAC} (Gestor Académico de Calendarios), centraliza y mantiene actualizada toda la información relevante sobre la planificación docente, además de aportar una serie de herramientas que permiten a los alumnos la planificación automática de su selección de matrícula reduciendo significativamente el tiempo que les tomaría hacerlo de forma manual y sin errores.\newline
	

\cleardoublepage

\begin{center}
	{\large\bfseries Same, but in English}\\
\end{center}
\begin{center}
	José Pablo Márquez Megías\\
\end{center}
\vspace{0.5cm}
\noindent\textbf{Keywords}: \textit{open source}, \textit{timetabling}, \textit{SOA}, \textit{API}, \textit{web scraping}
\vspace{0.7cm}

\noindent\textbf{Abstract}\\


\cleardoublepage

\thispagestyle{empty}

\noindent\rule[-1ex]{\textwidth}{2pt}\\[4.5ex]

D. \textbf{Ángel Ruíz Zafra} y Dª. \textbf{Kawtar Benghazi Akhlaki Sekkate}, Profesores del departamento de Lenguajes y Sistemas Informáticos.
\vspace{0.5cm}

\textbf{Informamos:}

\vspace{0.5cm}

Que el presente trabajo, titulado \textit{\textbf{Servicio web para la gestión de calendarios académicos}},
ha sido realizado bajo nuestra supervisión por \textbf{José Pablo Márquez Megías}, y autorizamos la defensa de dicho trabajo ante el tribunal
que corresponda.

\vspace{0.5cm}

Y para que conste, expiden y firman el presente informe en Granada a septiembre de 2024.

\vspace{1cm}

\textbf{El/la director(a)/es: }

\vspace{5cm}

\noindent \textbf{Ángel Ruíz Zafra} y \textbf{Kawtar Benghazi Akhlaki Sekkate}

\chapter*{Agradecimientos}






